\chapter*{Resumen}

En los últimos años, la investigación en Visión Artificial para que las máquinas puedan percibir el mundo físico que les rodea, al igual que hace el ser humano mediante la vista, ha experimentado un gran desarrollo. En este aspecto, el uso de arquitecturas neuronales profundas cuyo aprendizaje está dirigido por conjuntos de datos representativos de la tarea a abordar, ha permitido mejorar las prestaciones de los algoritmos tradicionales. De las tres tareas que pueden realizar las estructuras neuronales, detección, clasificación y predicción, la predicción visual es la menos común. Además, dicha tarea tiene un largo recorrido en su investigación y una gran utilidad. Es por ello que este trabajo fin de máster aborda la tarea de predicción haciendo uso de secuencias de imágenes con un elemento móvil.\\

Se ha realizado un estudio sobre cómo distintas estructuras neuronales, de naturaleza recurrente y no recurrente, pueden utilizarse para realizar la predicción de fotogramas en una secuencia de vídeo, donde las correlaciones espacio-temporales entre imágenes son importantes. Para ello, se ha creado una serie de secuencias sintéticas formadas por fotogramas en los que un único píxel activo se desplaza siguiendo una determinada dinámica temporal: lineal, parabólica o sinusoidal. Así mismo, se han considerado los fotogramas en dos formatos distintos, uno modelado, que reduce toda la imagen a las posiciones (\textit{x}, \textit{y}) del píxel activo, y otro crudo, que representa la imagen como una matriz 2D de píxeles. Para obtener las secuencias sintéticas,  se ha desarrollado un generador que permite crear ejemplos adaptados a un tipo de estudio concreto.\\

Con la realización de esta investigación se ha comprobado que, bajo determinadas hipótesis, es posible predecir satisfactoriamente la posición del objeto móvil en secuencias de imágenes. Además, la recurrencia en las estructuras neuronales aporta un gran valor para abordar este tipo de tarea, pues permite capturar la correlación temporal existentes en la secuencia.