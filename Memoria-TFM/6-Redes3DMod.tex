\chapter{Predicción de imágenes modeladas}\label{cap.redes3dmod}

En este capítulo se presentan todos los estudios realizados en cuanto a la predicción en imágenes modeladas, mejorando las redes en distintos factores según los resultados que se van obteniendo con el objetivo de encontrar la estructura más adecuada para predecir en el mundo real.\\

En el mundo de las imágenes modeladas se ha afrontado la predicción como un problema de regresión, en el que la entrada es el conjunto de instantes de tiempo que se consideran conocidos (\textit{n}\_\textit{points}) con sus pares de posiciones (\textit{x}, \textit{y}) y la salida es el instante de tiempo futuro con su par de coordenadas correspondiente. Cada uno de estas coordenadas pueden tomar cualquier valor numérico decimal por lo que, como se explicó en la Sección~\ref{sec.eval}, es necesario realizar un redondeo para obtener la posición final, pues los píxeles están siempre representados por números enteros.\\

A continuación se realiza el recorrido por las distintas estructuras entrenadas, presentando los resultados obtenidos y las conclusiones que dan lugar a la exploración de distintas vías para la mejora de los mismos.

\section{Redes no recurrentes}

\section{Redes recurrentes}
\subsection{Red básica}
\subsection{Aumento de neuronas}
\subsection{Aumento de capas}
\subsection{Red avanzada: Dinámica combinada}
