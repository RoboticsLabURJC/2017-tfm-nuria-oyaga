\chapter{Evaluación y métricas}\label{cap.evaluacion}

En este capítulo se explican las distintas métricas que se calculan tras pasar el conjunto de \textit{test} por la red previamente entrenada, permitiendo obtener una medida objetiva de la bondad de dicha red. Además se hace un análisis en profundidad del código Python desarrollado para tal fin y se hace una interpretación sobre los gráficos y resultados que arroja dicho código al terminar la ejecución.

\section{Métricas}

Para poder evaluar la distin

\section{Herramienta de evaluación}

En la Figura~\ref{fig.flujo_test} se puede visualizar el flujo que sigue el código para la evaluación de una red en concreto con un conjunto de \textit{test} determinado.

\vspace{10pt}
\begin{figure}[H]
    \begin{center}
        \begin{tikzpicture}[node distance=2cm]
            \node (ld) [treenodelong] {Lectura de datos};
            \node (net) [treenodelong, below of=ld] {Carga de la red};
            \node (pred) [treenodelong, below of=net] {Predicción};
            \node (trans) [treenodelong, below of=pred] {Transformación de posiciones};    
            \node (calc) [treenodelong, below of=trans] {Cálculo de métricas};
            \node (pres) [treenodelong, below of=calc] {Presentación de resultados};
            
            \draw [arrow] (ld) -- (net);
            \draw [arrow] (net) -- (pred);
            \draw [arrow] (pred) -- (trans);
            \draw [arrow] (trans) -- (calc);
            \draw [arrow] (calc) -- (pres);
        \end{tikzpicture}
        \caption{Diagrama de flujo del evaluador.}
	    \label{fig.flujo_test}
	\end{center}
\end{figure}

\section{Interpretación de resultados}